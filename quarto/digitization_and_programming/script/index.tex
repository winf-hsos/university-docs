% Options for packages loaded elsewhere
\PassOptionsToPackage{unicode}{hyperref}
\PassOptionsToPackage{hyphens}{url}
\PassOptionsToPackage{dvipsnames,svgnames,x11names}{xcolor}
%
\documentclass[
  letterpaper,
  DIV=11]{scrartcl}

\usepackage{amsmath,amssymb}
\usepackage{iftex}
\ifPDFTeX
  \usepackage[T1]{fontenc}
  \usepackage[utf8]{inputenc}
  \usepackage{textcomp} % provide euro and other symbols
\else % if luatex or xetex
  \usepackage{unicode-math}
  \defaultfontfeatures{Scale=MatchLowercase}
  \defaultfontfeatures[\rmfamily]{Ligatures=TeX,Scale=1}
\fi
\usepackage{lmodern}
\ifPDFTeX\else  
    % xetex/luatex font selection
\fi
% Use upquote if available, for straight quotes in verbatim environments
\IfFileExists{upquote.sty}{\usepackage{upquote}}{}
\IfFileExists{microtype.sty}{% use microtype if available
  \usepackage[]{microtype}
  \UseMicrotypeSet[protrusion]{basicmath} % disable protrusion for tt fonts
}{}
\makeatletter
\@ifundefined{KOMAClassName}{% if non-KOMA class
  \IfFileExists{parskip.sty}{%
    \usepackage{parskip}
  }{% else
    \setlength{\parindent}{0pt}
    \setlength{\parskip}{6pt plus 2pt minus 1pt}}
}{% if KOMA class
  \KOMAoptions{parskip=half}}
\makeatother
\usepackage{xcolor}
\setlength{\emergencystretch}{3em} % prevent overfull lines
\setcounter{secnumdepth}{-\maxdimen} % remove section numbering
% Make \paragraph and \subparagraph free-standing
\makeatletter
\ifx\paragraph\undefined\else
  \let\oldparagraph\paragraph
  \renewcommand{\paragraph}{
    \@ifstar
      \xxxParagraphStar
      \xxxParagraphNoStar
  }
  \newcommand{\xxxParagraphStar}[1]{\oldparagraph*{#1}\mbox{}}
  \newcommand{\xxxParagraphNoStar}[1]{\oldparagraph{#1}\mbox{}}
\fi
\ifx\subparagraph\undefined\else
  \let\oldsubparagraph\subparagraph
  \renewcommand{\subparagraph}{
    \@ifstar
      \xxxSubParagraphStar
      \xxxSubParagraphNoStar
  }
  \newcommand{\xxxSubParagraphStar}[1]{\oldsubparagraph*{#1}\mbox{}}
  \newcommand{\xxxSubParagraphNoStar}[1]{\oldsubparagraph{#1}\mbox{}}
\fi
\makeatother


\providecommand{\tightlist}{%
  \setlength{\itemsep}{0pt}\setlength{\parskip}{0pt}}\usepackage{longtable,booktabs,array}
\usepackage{calc} % for calculating minipage widths
% Correct order of tables after \paragraph or \subparagraph
\usepackage{etoolbox}
\makeatletter
\patchcmd\longtable{\par}{\if@noskipsec\mbox{}\fi\par}{}{}
\makeatother
% Allow footnotes in longtable head/foot
\IfFileExists{footnotehyper.sty}{\usepackage{footnotehyper}}{\usepackage{footnote}}
\makesavenoteenv{longtable}
\usepackage{graphicx}
\makeatletter
\newsavebox\pandoc@box
\newcommand*\pandocbounded[1]{% scales image to fit in text height/width
  \sbox\pandoc@box{#1}%
  \Gscale@div\@tempa{\textheight}{\dimexpr\ht\pandoc@box+\dp\pandoc@box\relax}%
  \Gscale@div\@tempb{\linewidth}{\wd\pandoc@box}%
  \ifdim\@tempb\p@<\@tempa\p@\let\@tempa\@tempb\fi% select the smaller of both
  \ifdim\@tempa\p@<\p@\scalebox{\@tempa}{\usebox\pandoc@box}%
  \else\usebox{\pandoc@box}%
  \fi%
}
% Set default figure placement to htbp
\def\fps@figure{htbp}
\makeatother

\KOMAoption{captions}{tableheading}
\makeatletter
\@ifpackageloaded{caption}{}{\usepackage{caption}}
\AtBeginDocument{%
\ifdefined\contentsname
  \renewcommand*\contentsname{Inhaltsverzeichnis}
\else
  \newcommand\contentsname{Inhaltsverzeichnis}
\fi
\ifdefined\listfigurename
  \renewcommand*\listfigurename{Abbildungsverzeichnis}
\else
  \newcommand\listfigurename{Abbildungsverzeichnis}
\fi
\ifdefined\listtablename
  \renewcommand*\listtablename{Tabellenverzeichnis}
\else
  \newcommand\listtablename{Tabellenverzeichnis}
\fi
\ifdefined\figurename
  \renewcommand*\figurename{Abbildung}
\else
  \newcommand\figurename{Abbildung}
\fi
\ifdefined\tablename
  \renewcommand*\tablename{Tabelle}
\else
  \newcommand\tablename{Tabelle}
\fi
}
\@ifpackageloaded{float}{}{\usepackage{float}}
\floatstyle{ruled}
\@ifundefined{c@chapter}{\newfloat{codelisting}{h}{lop}}{\newfloat{codelisting}{h}{lop}[chapter]}
\floatname{codelisting}{Listing}
\newcommand*\listoflistings{\listof{codelisting}{Listingverzeichnis}}
\makeatother
\makeatletter
\makeatother
\makeatletter
\@ifpackageloaded{caption}{}{\usepackage{caption}}
\@ifpackageloaded{subcaption}{}{\usepackage{subcaption}}
\makeatother

\ifLuaTeX
\usepackage[bidi=basic]{babel}
\else
\usepackage[bidi=default]{babel}
\fi
\babelprovide[main,import]{ngerman}
% get rid of language-specific shorthands (see #6817):
\let\LanguageShortHands\languageshorthands
\def\languageshorthands#1{}
\ifLuaTeX
  \usepackage[german]{selnolig} % disable illegal ligatures
\fi
\usepackage{bookmark}

\IfFileExists{xurl.sty}{\usepackage{xurl}}{} % add URL line breaks if available
\urlstyle{same} % disable monospaced font for URLs
\hypersetup{
  pdftitle={Vorwort \{.unnumbered\}},
  pdfauthor={Prof.~Dr.~Nicolas Meseth},
  pdflang={de},
  colorlinks=true,
  linkcolor={blue},
  filecolor={Maroon},
  citecolor={Blue},
  urlcolor={Blue},
  pdfcreator={LaTeX via pandoc}}


\title{Vorwort \{.unnumbered\}}
\usepackage{etoolbox}
\makeatletter
\providecommand{\subtitle}[1]{% add subtitle to \maketitle
  \apptocmd{\@title}{\par {\large #1 \par}}{}{}
}
\makeatother
\subtitle{Vorlesungsskript Digitalisierung und Programmierung}
\author{Prof.~Dr.~Nicolas Meseth}
\date{}

\begin{document}
\maketitle

\renewcommand*\contentsname{Inhaltsverzeichnis}
{
\hypersetup{linkcolor=}
\setcounter{tocdepth}{3}
\tableofcontents
}

\subsection{Warum dieses Buch?}\label{warum-dieses-buch}

Dieses Buch entstand aus der Erkenntnis, dass viele klassische
Lehrbücher der Informatik für Anfänger oft zu technisch und abstrakt
sind. In meiner langjährigen Lehrtätigkeit habe ich beobachtet, dass
Studierende besonders dann erfolgreich lernen, wenn sie die Konzepte der
Informatik in einem praktischen Kontext erleben können. Deshalb habe ich
mich entschieden, einen praxisorientierten Ansatz zu wählen, der
theoretische Grundlagen mit einem konkreten Projekt verbindet.

\subsection{Wen möchte ich wie
ansprechen?}\label{wen-muxf6chte-ich-wie-ansprechen}

Ich richte dieses Buch an Menschen ohne Vorkenntnisse in den Bereichen
Digitalisierung, Computertechnik oder Programmierung. Ich gehe nicht
davon aus, dass die Leserinnen und Leser eine besondere Motivation
mitbringen, sich mit diesen Themen zu beschäftigen. Wenn doch, umso
besser. Diese beiden Annahmen -- fehlende Vorkenntnisse und Motivation
-- treffen auf den Großteil meiner Studierenden zu, für die ich dieses
Buch in erster Linie geschrieben habe.

Mein Ziel ist es, sowohl die Kenntnisse als auch die Begeisterung für
die Informatik zu steigern -- zumindest bei einigen, die dieses Buch zur
Hand nehmen (müssen). Um dies zu erreichen, habe ich mich entschieden,
einen anderen Weg einzuschlagen als klassische Informatik-Lehrbücher:

\begin{itemize}
\item
  Ich verwende bewusst eine \textbf{einfache Sprache}. Das heißt nicht,
  dass wir keine Fachbegriffe einführen werden. Wir erklären die
  Sachverhalte aber zunächst in einer für alle Studierenden
  verständlichen Sprache und führen Fachbegriffe schrittweise ein.
\item
  Neben der Sprache knüpfe ich bei den Beispielen gezielt an bestehendes
  Wissen an. Ich verwende \textbf{Beispiele und Analogien aus dem
  Alltag}, um Ideen und Konzepte der Informatik zu veranschaulichen. Das
  mag nicht immer perfekt gelingen -- aber wenn es gelingt, hilft es
  meiner Erfahrung nach dabei, neue Themen besser zu verstehen.
\item
  Mein Fokus liegt auf dem \textbf{Verständnis} der Konzepte statt auf
  technischen Details. Ich verzichte bewusst auf zu viel Tiefe zugunsten
  eines zugänglichen Buches, das einen guten Überblick vermittelt und
  echtes Verständnis ermöglicht. Wer anschließend Lust auf mehr Tiefe
  hat, bekommt von mir in jedem Kapitel Leseempfehlungen an die Hand.
\item
  Ich bin überzeugt, dass konkrete Projekte das Interesse und
  Verständnis am besten fördern. Dieses Buch verbindet das
  \textbf{LiFi-Projekt} mit den theoretischen Grundlagen der Informatik
  und führt Schritt für Schritt an algorithmisches Denken und
  Programmierung heran. Das Ergebnis ist ein fertiges Produkt, für das
  die Leserinnen und Leser alle erlernten Kenntnisse praktisch anwenden
  mussten -- ganz nach dem Prinzip „Learning by Doing''.
\item
  Mir ist bewusst, dass viele der jüngeren Generation das Lesen eines
  Buches als Herausforderung empfinden. Dennoch halte ich Bücher für
  unverzichtbar, um komplexe Themengebiete zu erschließen. Um den
  Leseprozess zu erleichtern, stelle ich ergänzende \textbf{Videos und
  Audioaufnahmen} bereit, die in den Kapiteln verlinkt und über QR-Codes
  zugänglich sind.
\end{itemize}

\subsection{Wie ist das Buch
aufgebaut?}\label{wie-ist-das-buch-aufgebaut}

Dieses Buch befasst sich mit der Frage, wie wir Computer zum Lösen von
Problemen einsetzen können. Das Schaubild in
Abbildung~\ref{fig-advance-organizer-dap} visualisiert die wichtigsten
Themenblöcke.

\begin{figure}

\centering{

\includegraphics[width=0.6\linewidth,height=\textheight,keepaspectratio]{index_files/mediabag/advance_organizer_da.png}

}

\caption{\label{fig-advance-organizer-dap}Überblick über die
Themenblöcke dieses Buches.}

\end{figure}%

Ich orientiere mich übergeordnet an dem Thema des Problemlösens
(\emph{problem-solving}), das sich in zwei Bereiche gliedert:

\begin{enumerate}
\def\labelenumi{\arabic{enumi}.}
\item
  \textbf{Algorithmen} (\emph{algorithms}) bilden den Kern des
  Problemlösens und der Informatik. Sie beschreiben die notwendigen
  Schritte zur Lösung eines Problems.
\item
  \textbf{Kommunikation} (\emph{communication}) umfasst das Teilen von
  Informationen in Computernetzwerken. Mit der weiten Verbreitung des
  Internets ist dieser Aspekt zentral für die moderne Computernutzung
  geworden. Lösungen, die ein Computer erzeugt, können heute unmittelbar
  über Netzwerke weltweit geteilt werden.
\end{enumerate}

Im Bereich der Algorithmen beschäftigen wir uns zunächst damit, wie wir
Probleme für Computer verständlich und lösbar beschreiben können. Eine
zentrale Rolle spielt dabei die \textbf{Informationsrepräsentation
}(\emph{information representation}): Wie können wir Zahlen, Texte,
Bilder, Videos, Audioaufnahmen und andere wichtige Inhalte so
darstellen, dass ein Computer damit arbeiten kann?

Nachdem wir das verstanden haben, widmen wir uns der
\textbf{Informationsverarbeitung }(\emph{information processing}) --
also der Frage, wie Eingabeinformationen so verarbeitet werden können,
dass eine Lösung entsteht. Dies ist die Kernaufgabe der Algorithmen, und
wir untersuchen, wie ein Algorithmus sowohl für Menschen als auch für
Computer verständlich dargestellt werden kann. Dabei lernen wir die
Programmiersprache Python kennen, die als moderne Programmiersprache
Algorithmen in ausführbare \textbf{Programme} (\emph{programs})
übersetzt. Anhand einfacher Beispiele wie der Addition zweier Zahlen
verstehen wir zudem, wie die Ausführung von Programmierbefehlen im
Rechner auf der Ebene der Bits funktioniert.

Bei der Kommunikation stellen wir uns die Frage, wie \textbf{das Teilen
von Informationen} (\emph{information sharing}) über unterschiedliche
Medien wie Kabel, Luft oder Licht funktioniert. Wir lernen dabei etwas
über das Senden und Empfangen von Signalen, über Protokolle -- also
Vereinbarungen zur Informationsübermittlung -- sowie über die Frage, wie
wir unsere Kommunikation effizient und gleichzeitig sicher gestalten
können.

\subsection{Wie sollte man dieses Buch
lesen?}\label{wie-sollte-man-dieses-buch-lesen}

Das Buch ist für eine lineare Lektüre von vorne nach hinten konzipiert.
An der Hochschule Osnabrück behandeln wir in der zugehörigen
Veranstaltung pro Woche ein Kapitel -- gelegentlich auch zwei, abhängig
von der Verteilung der Feiertage im Semester. Die Kapitel hängen
zusammen und bauen teilweise aufeinander auf.

Wie beschrieben orientiert sich dieses Buch an der Durchführung eines
Projekts -- dem LiFi-Projekt. Das Projekt bildet den Ausgangspunkt jedes
Kapitels, und für jedes Thema wird der praktische Bezug zum Projekt
hergestellt. Was genau das LiFi-Projekt beinhaltet, schauen wir uns im
nächsten Kapitel an.




\end{document}
